\documentclass[12pt,oneside]{article}

\setlength{\parindent}{0in}

\begin{document}

Chad Reynolds \hfill 
CS:5850 Programming Language Foundations \\ 
Final Project Proposal \hfill 
Due:  November 15, 2016 \\


\section*{Proposal} 
My plan is to work on the suggested "Implementing pure
lambda calculus" project.  My implementation would be done in Haskell, encoding
the lambda terms as a Haskell datatype and using De Bruijn indices to avoid
variable capture.  The output would be some form of \LaTeX{} markup, showing 
syntax trees or the terms (or both!).  My plan for parsing input is to use the 
Haskell Parsec library, which simplifies the creation of simple 
language parsers.  


\section*{Preferred Presentation Dates} 
\begin{enumerate}
    \item Dec 1
    \item Dec 8
    \item Dec 6
\end{enumerate}

\section*{Getting Started}
To get started on my project, I have explored both how to implement a simple 
parser in Haskell and methods for implementing De Bruijn indices.  I installed 
the Parsec library and looked over code for implementing a parser for a 
basic calculator language with it.  Along with this, I have been considering 
how to implement reduction in a way to facilitate showing each term from 
beginning to normalized.


\end{document}
